\documentclass[a4paper,12pt]{article}
\usepackage[english]{babel}
\usepackage[utf8]{inputenc}
\usepackage{setspace}

% Larger borders -- we do not want do waste paper, even if it is only paper on screen =)
\usepackage[top=2.5cm, bottom=2.5cm, left=2cm, right=2cm]{geometry}
% Remove auto indentation of paragraphs.
\setlength\parindent{0pt}

% Palatino font (nicer serif font: Times is for oldies)
%\renewcommand*\rmdefault{ppl}

% Nested itemize list bullet style
\renewcommand{\labelitemi}{$\bullet$}
\renewcommand{\labelitemii}{$\circ$}
\renewcommand{\labelitemiii}{--}

% Math packages
\usepackage{mathtools}
\usepackage{amsmath}
\usepackage{amsfonts}
\usepackage{amssymb}

% Graphic packages
\usepackage{graphicx}
\usepackage{subcaption}
\usepackage{float}
\usepackage{adjustbox}
\usepackage{tikz}
\usepackage{forest,array}
\usetikzlibrary{shadows}

% Graphs styles
\forestset{
  giombatree/.style={
    for tree={
      grow = east,
      parent anchor=east,
      child anchor=west,
      edge={rounded corners=2mm},
      fill=violet!5,
      drop shadow,
      l sep=10mm,
      edge path={
        \noexpand\path [draw, \forestoption{edge}] (!u.parent anchor) -- +(5mm,0) -- (.child anchor)\forestoption{edge label};
      }
    }
  }
}
\forestset{
  qtree/.style={
    for tree={
      parent anchor=south,
      child anchor=north,
      align=center,
      edge={rounded corners=2mm},
      fill=violet!5,
      drop shadow,
      l sep=10mm,
    }
  }
}

% BAN logic macros
\newcommand{\believes}{\mid\!\equiv}
\newcommand{\sees}{\triangleleft}
\newcommand{\oncesaid}{\mid\!\sim}
\newcommand{\controls}{\Rightarrow}
\newcommand{\fresh}[1]{\#(#1)}
\newcommand{\combine}[2]{{\langle #1 \rangle}_{#2}}
\newcommand{\encrypt}[2]{{ \{ #1 \} }_{#2}}
\newcommand{\sharekey}[1]{\xleftrightarrow{#1}}
\newcommand{\pubkey}[1]{\xmapsto{#1}}
\newcommand{\secret}[1]{\xleftrightharpoons{#1}}

\newcommand{\projectname}{Aeronautical Communication System}
\newcommand{\projectnameabbr}{ACS}

% Hides ugly links from the index
\usepackage[hidelinks]{hyperref}
% Landscape format pdf pagess
\usepackage{pdflscape}


\begin{document}
\pagenumbering{gobble}

{\setstretch{1.0}
  \begin{titlepage}
  	\centering
  	\includegraphics[width=6cm]{img/unipi.pdf}\par
    \vspace{1.5cm}
    {\Large Department of Information Engineering \par}
  	\vspace{1.5cm}
  	{\huge\textsc{\projectname{}}\par}
    \vspace{0.5cm}
    {\Large Performance Evaluation of Computer Systems and Networks project \par}
  	\vspace{2cm}
  	Pietro \textsc{Gronchi}\par
  	Amedeo \textsc{Pochiero}\par
    Giovan Battista \textsc{Rolandi}

  	\vfill

    % Bottom of the page
  	{\large A.Y. 2019-2020\par}
  \end{titlepage}
}


\clearpage
\tableofcontents
\clearpage
\pagenumbering{arabic}

\section{Introduction}
\projectname{} is a cummunication system between aircrafts (AC) and a control tower (CT).
Connection between AC and CT is provided by ground base stations (BS), which are deployed over a grid, at a distance M from neighbors.

\begin{figure}[H]
  \centering
  \includegraphics[width=6cm]{img/grid.pdf}
  \caption{System topology}
  \label{fig:grid}
\end{figure}

Deployment grid has a square topology, eg. number of rows is equal to number of columns.
In Figure~\ref{fig:grid}, deployment grid is drawn with dotted blue lines, while solid black lines delimit the area, called \emph{cell}, where an AC is closer to the inner base station.
$L$ is a shortcut to indicate $M \cdot n$, where $n$ is the number of BSs that lay on a row (or a column).

Each AC selects only one BS at a time as its serving BS, and can transmit only one packet at a time.
Possibly, packets are backlogged in a queue on the AC.

ACs execute the handover procedure every $t$ seconds, accordingly with the following algorithm:
\begin{verbatim}
if (serving BS is the closest one) then
  do nothing
else
  select closest BS as serving BS
  pause transmissions for penalty time (p)
endif
\end{verbatim}

\section{Objectives}
Minimize end to end delay between AC and CT, using the maximum packet interarrival frequency, without destabilizing queues in the system.

\section{Performance Indexes}
\begin{itemize}
  \item response time
  \item queue length
\end{itemize}

\section{Scenario}
\begin{itemize}
  \item \textbf{k}: it represents the interarrival time of packets;
  \item \textbf{M}: it represents the distance between two consecutive BS on the same row, or column, and it is fixed at 25 km;
  \item \textbf{d}: it represents the distance between AC and BS;
  \item \textbf{s}: it represents the service time of a packet, and is given by $s = T \cdot d^{2}$;
  \item \textbf{T}: it is a tuning parameter for service time, whose value is costant for every AC in the system;
  \item \textbf{t}: it represents the handover period;
  \item \textbf{p}: it represents the handover penalty;
\end{itemize}

\section{Calibration}
% TODO Roberto 3.1.1 -- Parameter analysis
% and always respect the stability condition




\section{Model}
The ACS can be modeled with the following system from queueing theory.

\begin{figure}[H]
  \centering
  \includegraphics{img/model.pdf}
  \caption{M/G/1 system for ACS}
  \label{fig:model}
\end{figure}

\section{Simulator}
\subsection{Software}
Simulator has been coded in C++ using the OMNeT++ environment and the INET framework.
\begin{figure}[H]
  \centering
  \includegraphics{img/aircraft-ned.png}
  \caption{Module structure of Aircraft}
  \label{fig:aircraft-ned}
\end{figure}
% TODO inserire qui descrizione del disegnino del modulo AC, con dentro Transmitter e Mobility Module

\subsection{Movement of AC}
Movement of AC is given by the following algorithm, implemented using the \texttt{TurtleMobility} module from INET:
\begin{itemize}
  \item generate a vector $\vec{s}$ such that $|\vec{s}| \in \mathcal{U}(0, M_{T}\sqrt{2})$ and $\angle{s} \in \mathcal{U}(0, 2\pi)$; upper bound of $\vec{s}$ is maximum possibile distance before AC surely wraps around;
  \item given that AC is in position $\vec{a}$, move to position $\vec{a} + \vec{s}$, at constant $v$ speed, possibly wrapping around at the edges of the simulated topology;
  \item once reached new position, re-do from the beginning;
\end{itemize}

This algorithm is equivalent to \texttt{Random WayPoint}, but takes in account the possibility of a wrap around.

\section{Verification}
In order to verify the correctness of our code, we recorded the service time for a simulation and looked at a small time interval in order to spot abviously wrong situations.

\begin{figure}[H]
  \centering
  \includegraphics[width=\textwidth]{img/verification-via-service-time.pdf}
  \caption{Service time during a test simulation}
  \label{fig:verification-via-service-time}
\end{figure}

We can see that:
\begin{itemize}
  \item service time has a paraboloid shape, which is expected since it is proportional with $d^2$, which $d$ increases linearly during time, because AC travel at costant speed;
  \item graph discontinuity, like the one between 18950 and 19000, represent the penalty time, which can be observed after an handover; service time correctly decreases after the handover;
  \item graph non derivability, like two points slightly before 19100 and 19150, represent the moment when AC randomly changes direction;
\end{itemize}

At this point, we are confident that the code for our simulator is correctly modeling the real world scenario, thus we can go through a more in-depth inspection by performing some validation test.

\section{Validation tests}
\subsection{Considerations on distance}
\begin{figure}[H]
  \centering
  \begin{subfigure}[b]{0.45\textwidth}
    \centering
    \includegraphics{img/dmax.pdf}
    \caption{Computation of $d_{max}$}
    \label{fig:dmax}
  \end{subfigure}
  ~
  \begin{subfigure}[b]{0.45\textwidth}
    \centering
    \includegraphics{img/d-simplified.pdf}
    \caption{Simplified distance computation}
    \label{fig:d-simplified}
  \end{subfigure}
  \caption{Computation of distance between AC and BS}
  \label{fig:d}
\end{figure}

Service time depends on distance between AC and BS.
Distance between AC and BS, namely \emph{d}, belongs to interval $[ 0, d_{max} [$ where $d_{max}$ can be computed as $d_{max\_in} + d_{max\_out}$, where:
\begin{itemize}
  \item $d_{max\_in} = \frac{M \sqrt{2}}{2} $
  \item $d_{max\_out} = vt$
\end{itemize}

What is the worst case? Referring to Figure~\ref{fig:dmax}, let's assume that an AC is moving in direction $BS_g \rightarrow B$, and that, at a given time $t_g$, $\overrightarrow{AC} \in cell$ and $\overrightarrow{AC} - \overrightarrow{A} < \epsilon$ with $\epsilon$ small as you wish, eg. AC is at point A but still inside the cell where the nearest BS is $BS_g$; and let's assume that, at this time, AC initiates the handover procedure and discovers that the nearest BS is $BS_g$. Immediately after this handover procedure, performed without any penalty time, AC is out of the cell and will not initiate the handover procedure before $t$ time: this means that, when it is in point B, it has traveled for, $d_{max\_out} = v t$, since the last handover procedure, and that the longest distance $d_{max}$ between AC and BS is given by:

$$ d_{max} = \frac{M \sqrt{2}}{2} + vt $$

This result can be used to validate our simulator.

\subsection{Simplification on distance}
In order to compute the distribution of service time, and to validate our model against a known queuing theory model, we had to make some simplifications for the computation of distance, and in particular: a single circle area around a single BS was used, as shown in Figure~\ref{fig:d-simplified}.

This way, we did not take in account for:
\begin{itemize}
  \item handover, because its probability is difficult to compute;
  \item area near cell's corners (coloured area), because it introduced a non-trivial-to-manage discontinuity in our formulas;
\end{itemize}

We want to compute $P\{D = x\}$, eg. probability that, picked a random position for AC inside the cell, distance $D$ between AC and BS equals to $x$.
Intuition suggests that, with bigger circumferences (centered in BS) with radius $x$, in the real world, more points are suitable to satisfy $D = x$, thus $P\{D = x\}$ is proportional with circumference, which is proportional with radius.
So, a reasonable PDF for D is:

$$ f_D(x) = k \cdot 2 \pi x, 0 \leq x \leq \frac{M}{2} $$

with $k = 2.03718 \times 10^{-9}$ (computed with normalization condition).

Since our goal is computing the distribution for service time, we note that:

$$ F_D(x) = \int_{0}^{x} f_D(x) dx = k \pi x^2 $$

and also that

$$ s = T \cdot x^2 \implies x = + \sqrt{\frac{s}{T}} $$

and

$$ F_S(s) = F_D\left(\sqrt{\frac{s}{T}}\right) = \frac{k \pi}{T} s $$

then

$$ f_S(s) = \frac{k \pi}{T}, s \in \left[0; \frac{TM^2}{4}\right] \implies S \in \mathcal{U}\left(0, \frac{TM^2}{4} \right)$$

This result can be used to validate our simulator.
%% TODO which we can fit?
% we ran our simulation with 1 BS for 1M times, then deleted all distances above M/2 and fitted that distribution,
% and we saw that is was there :-) QUESTA COSA VA FATTA NEI VALIDATION EXPERIMENTS

\subsection{Handover convenience}
Last test was on the handover convenience: it is straightforward that we want

$$ s[t] \geq s[t + 1] $$

for each $t, t + 1$ such that and handover has been performed in between, eg. service time never increases after an handover.

This result can be used to validate our simulator.

\section{Validation experiments}









\section{Experiments}
% TODO da fare dopo che il simulatore sarà stato correttamente validato
% TODO steady sate probability: ci serve un warmup time? Roberto 4.1, figura 8

\end{document}
